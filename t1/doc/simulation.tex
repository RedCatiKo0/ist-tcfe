\section{Simulation Analysis}
\label{sec:simulation}

In table (\ref{tab:sim}) we can find the values obtained by the simulation.

\begin{table}[hbt!]
    \centering
    \begin{tabular}{|l|r|}
    \hline    
    {\bf Quantity} & {\bf Value [A or V]} \\ \hline
    \input{op_tab}
    \end{tabular}
    \caption{Simulation Results - Variables preceeded by @ are currents in Ampere, while 
    all the other variables are voltages in Volt}
    \label{tab:sim}
\end{table} 

We can see that the values obtained from both theoretical 
methods (which gave identical results) are the same as the ones that result from the simulation.
It is worth noting that \emph{v(7a)} and \emph{v(7b)} are just auxilary voltages used to define $I_c$
in \emph{Ngspice} so that we can simulate the current dependent voltage source $V_c = K_c I_c$ and that
\emph{v(2,5)} corresponds to $V_b$ and \emph{i(vaux)} to $I_c$.