\section{Análise Teórica}
\label{sec:analysis}

% FALTAM AS TABELAS DE OCTAVE COM OS VALORES QUE SÃO PEDIDOS NO PP

\subsection{Estudo do \emph{Envelope Detector}}

Começámos por estudar o \emph{Envelope Detector} utilizado. Este é constituído por um \emph{Full-Wave Bridge Rectifier}
em paralelo com um condensador, com capacidade $C_1 = 3.1 \: \mu F$ e por uma resistência $R_1 = 4.9 k \Omega$
e tem como função aproximar a corrente alterna a corrente contínua.
\\
Como visto nas aulas teóricas, sabemos que a tensão de output de um \emph{Full-Wave Bridge Rectifier} corresponde ao módulo da tensão
da fonte alternada, que neste caso corresponde à tensão imposta no transformador, ou seja,
$v_R = |1 / 10 \: v_{AC}| = |A\: cos(\omega t)| :\ V$, em que $A = 23 \: V$, $\omega = 2 \pi f$ e $f = 50 \: Hz$,
onde supomos que a tensão imposta pelo transformador é superior a $V_{ON}$, ou seja, à tensão a partir da qual circula corrente através dos díodos 
- consideramos um modelo teórico dos díodos onde supomos que só circula corrente para $V > V_{ON}$.
\\
Assim, a tensão no output para $0 < t < \pi / 2 \omega$ é dada por $v_E = A cos(\omega t) \:V$,
onde supomos novamente que os díodos estão ligados. De seguida, apresentam-se um conjunto de expressões úteis para o
estudo do \emph{Envelope Detector}, onde $i_d$ é a corrente que passa pelos díodos, $i_R$ a corrente na resistência
e $i_C$ a corrente no condensador e onde se considerou o intervalo $0 < t < \pi / 2 \omega$ (cosseno positivo):

\begin{equation}
    i_d = i_R + i_C
\end{equation}

\begin{equation}
    i_C = C_1 \frac{dv_E}{dt} = - C_1 A \omega sin(\omega t)
\end{equation}

\begin{equation}
    i_R = \frac{v_E}{R_1}
\end{equation}

De forma semelhante ao estudo realizado em aula, procuramos descobrir o instante $t_{OFF}$ em que a corrente 
que passa pelo díodo é nula (díodo desligado). Utilizando as expressões anteriores e impondo a condição $i_d = 0$ obtém-se:

\begin{equation}
    t_{OFF} = \frac{1}{\omega} atan(\frac{1}{\omega R_1 C_1})
\end{equation}

Para $t > t_{OFF}$ o condensador descarrega sobre $R_1$ já que não passa corrente pelos díodos,
pelo que a voltagem de output do \emph{Envelope Detector} é dada pela expressão:

\begin{equation}
    v_E(t) = A cos(\omega t_{OFF}) e^{-\frac{t - t_{OFF}}{R_1 C_1}}
\end{equation}

O díodo volta a ligar-se quando a função dada pela expressão anterior interseta a voltagem normal do \emph{Full-Wave}.
Por fim, através das expressões anteriores e replicando o comportamento do circuito para 10 períodos,
efetuou-se um \emph{plot} da voltagem de output do \emph{Envelope Detector} que se encontra 
na figura (\ref{fig:TeoEnvelopeDetector}).

\begin{figure}[H] \centering
  \includegraphics[width=0.7\linewidth]{venvlope.eps}
  \caption{Voltagem no \textcolor{red}{\emph{Envelope Detector}} e no \textcolor{blue}{\emph{Full-Wave Bridge Rectifier}} - Análise Teórica}
  \label{fig:TeoEnvelopeDetector}
\end{figure}

Como é vísivel no gráfico da figura (\ref{fig:TeoEnvelopeDetector}), o \emph{Envelope Detector} transforma a corrente alternada em corrente
aproximadamente continua, verificando-se no entanto que existe ainda uma oscilação significativa da tensão - \emph{rippling}.
Este efeito será minimizado pelo \emph{Voltage Regulator} que estudamos de seguida.



\subsection{Estudo do \emph{Voltage Regulator}}

O \emph{Voltage Regulator} utilizado neste trabalho é constítuido por 18 díodos ligados em série - cujos modelos correspondem ao
\emph{default} do \emph{Ngspice}, ver secção (\ref{sec:simulation}) - e por uma resistência $R_1 = 4.9 \:k\Omega$, tendo como
objetivo atenuar o \emph{rippling} referido anteriormente.
\\
Analisou-se o \emph{Voltage Regulator} por análise incremental (\emph{Incremental Analysis}), dividindo o estudo da
voltagem de output, $v_O$, nas suas componentes DC, $V_O$, e AC, $v_o(t)$, de acordo com a expressão (\ref{eq:DCACAnalysis}).
É de notar que idealmente teríamos $V_O = 12 \: V$ e $v_o(t) = 0 \: V$.

\begin{equation}
    \label{eq:DCACAnalysis}
    v_O = V_O + v_o(t)
\end{equation}

De forma a realizar este tipo de análise, foi necessário determinar a resistência incremental dos díodos utilizados, $r_d$, bem 
como a voltagem $V_{ON}$, cujos valores se encontram na tabela (\ref{tab:SimValues}). Estes valores foram obtidos através de um
\emph{script} de \emph{Ngspice} e serão, portanto, discutidos na secção \ref{sec:simulation}.
\\
A componente DC foi obtida através de \emph{Operating Point Analysis}. Neste caso, os díodos correspondem a fontes de tensão
independentes em série com voltagem $V_{ON}$, pelo que consideramos que a componente DC de output do circuito corresponderá
a $V_O = n V_{ON}$.
\\
Na análise da componente AC, o problema reduz-se a um divisor de tensões:

\begin{equation}
    v_o(t) = \frac{n r_d}{n r_d + R} v_e(t)
\end{equation}

Onde $v_e(t)$ corresponde à componente AC do \emph{Envelope Detector}, ou seja, à voltagem de output do \emph{Envelope Detector}
subtraida pela sua média (componente DC).
\\
Assim, é de esperar que o output do conversor AC-DC desenvolvido seja dado por:

\begin{equation}
   v_O = n V_{ON} + \frac{n r_d}{n r_d + R} (v_E - V_E)
   \label{eq:OutputVoltageTeo}
\end{equation}

Os valores obtidos para a componente DC, $V_O$, e do ripple, $Ripple(v_O)$, da voltagem de output do \emph{Voltage Regulator} estão apresentados
na seguinte tabela (\ref{tab:TeoValues}):

\begin{table}[H]
    \centering
    \begin{tabular}{|l|r|}
    \hline
    \input{TeoIncrementalAnalysis.tex}
    \end{tabular}
    \caption{Valores obtidos através da análise teórica;}
    \label{tab:TeoValues}
  \end{table}

O \emph{plot} da voltagem de output do circuito, $v_O$, durante 10 períodos, corresponde à figura (\ref{fig:TeoVoltageRegulator}).

\begin{figure}[H] \centering
  \includegraphics[width=0.7\linewidth]{volreg.eps}
  \caption{Voltagem no \emph{Voltage Regulator} - Voltagem Output do Circuito - Análise Teórica}
  \label{fig:TeoVoltageRegulator}
\end{figure}

Podemos verificar uma diminuição de fenómenos de \emph{voltage rippling}, o que resulta do facto de $R >> r_d$ (ver equação (\ref{eq:OutputVoltageTeo})), pelo que a função
do \emph{voltage regulator} é assim cumprida.
\\
Por último, fez-se ainda um \emph{plot} do desvio da voltagem de output face aos $12 \: V$ esperados que se apresenta na figura (\ref{fig:TeoDeviation}).
Podemos verificar que a voltagem de output é bastante próxima de corrente DC com flutuações pequenas face aos 12 V.

\begin{figure}[H] \centering
  \includegraphics[width=0.7\linewidth]{outdev.eps}
  \caption{Desvio da voltagem de Output face aos 12 V - Simulação}
  \label{fig:TeoDeviation}
\end{figure}