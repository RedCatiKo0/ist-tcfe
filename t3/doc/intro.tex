\clearpage

\section{Introdução}
\label{sec:introduction}

O objetivo desta atividade laboratorial consiste na concepção de um circuito que transforme corrente alterna (AC - \emph{alternating current}) com
amplitude de tensão 230 V, em corrente contínua (DC - \emph{direct current}) de tensão 12 V  - \emph{AC-DC Converter}.
\\
O circuito desenvolvido é composto por um \emph{Envelope Detector} e um \emph{Voltage Regulator} e encontra-se representado na figura (\ref{fig:circuito}).
O funcionamento do mesmo é analisado através de modelos teóricos na secção (\ref{sec:analysis}), procurando-se estudar tanto o comportamento
do \emph{envelope detector} como o do \emph{voltage regulator}, onde para este efeito realizaram-se: gráficos relativos à voltagem de \emph{output} 
de cada um dos subcircuitos, um gráfico relativo ao desvio da voltagem do circuito face aos 12 V esperados e tendo-se ainda calculado o
\emph{voltage ripple} e a voltagem média aos terminais de saída do circuito.
\\
Na secção \ref{sec:simulation} procurou-se simular o \emph{AC-DC converter} efetuando-se os mesmos gráficos e cálculos feitos na análise teórica. 
Nesta secção compararam-se ainda os resultados obtidos através da simulação com os obtidos via métodos teóricos lado a lado.
Por outro lado realizou-se ainda o cálculo do mérito do circuito de acordo com os critérios definidos para este trabalho - secção (\ref{sec:Merit}).
\\
Por fim, na conclusão, secção (\ref{sec:conclusion}), resumem-se os resultados principais deste trabalho laboratorial e recapitulam-se as diferenças
entre os modelos teóricos e a simulação encontradas.

\begin{figure}[H] \centering
    \includegraphics[width=0.7\linewidth]{T3Circuit.pdf}
    \caption{Esquema do circuito utilizado.}
    \label{fig:circuito}
\end{figure}