\section{Conclusão}
\label{sec:conclusion}

Neste trabalho desenvolvemos um conversor AC-DC utilizando um \emph{Envelope Detector} e um \emph{Voltage Regulator} que, como verificámos
nas secções de análise teórica (\ref{sec:analysis}) e através da simulação (\ref{sec:simulation}), produz resultados satisfatórios.
\\
O estudo do circuito resultou no cálculo de várias grandezas (voltagem média de \emph{output} ou \emph{voltage ripple}) e \emph{plot}'s de gráficos que foram
obtidos através da simulação e da análise teórica. Estes resultados foram apresentados e discutidos nas respetivas secções 
e comparados na secção da simulação (\ref{sec:simulation}) tendo-se verificado pequenas diferenças entre o modelo teórico e a simulação do \emph{Ngspice},
principalmente no fenómeno de \emph{voltage rippling}. 
Estas diferenças deverão ser consequência da lineariedade dos modelos utilizados na secção teórica (modelo ideal do díodo e análise incremental)
que, apesar de apresentarem resultados muito satisfatórios (próximos dos da simulação), não são capazes de descrever a vertente não
linear do circuito com a mesma precisão que o modelo mais complexo utilizado pelo \emph{Ngpsice}.
\\
O mérito do trabalho foi calculado na secção (\ref{sec:Merit}), através da determinação do custo dos componentes utilizados no circuito e com os
valores médios da voltagem de \emph{output} e do \emph{voltage rippling} determinados através da simulação.
\\
Concluindo, consideramos que os objetivos do trabalho foram atingidos tendo sido possível desenvolver e estudar um conversor AC-DC tanto através
de métodos teóricos como através de simulações numéricas.
