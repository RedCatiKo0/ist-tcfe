\section{Simulação do Circuito}
\label{sec:simulation}

Nesta secção, simulámos o circuito que criámos, ilustrado na figura (\ref{fig:circuito}), usando o \emph{Ngspice}.
É de notar que a resistência entitulada de \emph{Load} não está efetivamente incluida no circuito simulado
e corresponde apenas à resistência de um aparelho genérico ligado à corrente DC. Para além disso, não se simulou o
transformador, colocando apenas uma fonte de tensão alternada, $v_{in}$, com a amplitude pretendida, $A = 23 \: V$,
considerando-se o número de espiras $n = 10$.
\\
A simulação realizada correspondeu a uma análise transiente do circuito durante 10 períodos, ou $200 \: ms$, com um
passo temporal de $\Delta t = 0.2 ms$, de modo a recolher $1000$ pontos. Esta permitiu
obter a voltagem média de \emph{output} e a \emph{voltage ripple}, onde este último valor corresponde à diferença entre os valores máximos
e mínimos da voltagem no \emph{output}. As grandezas em causa encontram-se apresentadas na tabela (\ref{tab:SimValues}).

\begin{table}[H]
    \centering
    \begin{tabular}{|l|r|}
    \hline    
    {\bf Quantidade} & {\bf Valor (V, $\Omega$)} \\ \hline
    \input{CircuitSim_tab}
    \end{tabular}
    \caption{Valores obtidos através da simulação de Ngspice; Por ordem: $V_O$, $ripple(v_O)$, $V_{ON}$ e $r_d$}
    \label{tab:SimValues}
\end{table}


Também se obtiveram os valores para $V_{ON}$ e $r_d$ que foram utilizados na secção (\ref{sec:analysis}. O primeiro foi considerado a média da voltagem medida
nos terminais de um diodo, neste caso $D_5$. O segundo foi calculado como sendo a razão das amplitudes de oscilação da voltagem medida
nos terminais de um diodo e da corrente que passa num diodo, neste caso $D_5$.
\\
Por comparação dos valores das tabelas (\ref{tab:TeoValues}) e (\ref{tab:SimValues}), nota-se que os valores da voltagem de output ($V_O$ ou mean(v(4,3))) são idênticos,
o que confirma que a voltagem de output oscila em torno desse valor, tal como previsto no método de \emph{incremental analysis}, utilizado na na secção \ref{sec:analysis}.
No entanto, apesar dos valores do ripple ($Ripple(v_O)$ ou (maximum(v(4,5)) - minimum(v(4,5))) / (maximum(i(Vaux)) - minimum(i(Vaux)))) serem consistentes já que apresentam
a mesma ordem de grandeza, têm um desvio percentual de $\epsilon_{Ripple(v_O)} = 53 \%$, o que poderá em parte decorrer de uma medição pouco rigorosa da resistência
incremental, $r_d$, mas dever-se-à também à utilização de um modelo linear na análise teórica que não reflete na totalidade o comportamento do díodo.
\\
\\
Da simulação também se obtiveram os gráficos (\ref{fig:SimEnvelopeDetector}) e (\ref{fig:SimVoltageRegulator}) correspondentes às voltagens de \emph{output}
no \emph{Envelope Detector} e no \emph{Voltage Regulator}.

\begin{figure}[H] \centering
  \includegraphics[width=0.7\linewidth]{envelope_detector_output.pdf}
  \caption{Voltagem no \emph{Envelope Detector} - Simulação}
  \label{fig:SimEnvelopeDetector}
\end{figure}

\begin{figure}[H] \centering
  \includegraphics[width=0.7\linewidth]{voltage_regulator_output.pdf}
  \caption{Voltagem no \emph{Voltage Regulator} - Voltagem Output do Circuito - Simulação}
  \label{fig:SimVoltageRegulator}
\end{figure}

Por fim, realizou-se ainda um \emph{plot} do desvio da voltagem de \emph{output} do circuito aos 12 V, ou seja,
de $v_O - 12$ de forma a verificar o quão próximo é a tensão à saída de uma tensão DC ideal. O gráfico em causa corresponde à figura (\ref{fig:SimDeviation}).

\begin{figure}[H] \centering
  \includegraphics[width=0.7\linewidth]{output_deviation.pdf}
  \caption{Desvio da voltagem de Output face aos 12 V - Simulação}
  \label{fig:SimDeviation}
\end{figure}

Por observação dos gráficos obtidos e por comparação com os gráficos da secção da análise teórica, secção (\ref{sec:analysis}), nota-se que estes são semelhantes,
nomeadamente que oscilam em torno de valores próximos, $V_E \approx 20 \: V$, no caso dos gráficos da voltagem no \emph{Envelope Detector}, e que são idênticos, $V_O \approx 12 \: V$,
no caso dos gráficos da voltagem no \emph{Voltage Regulator}.
\\
Refere-se novamente que as diferenças entre o modelo teórico e a simulação, neste caso o facto das amplitudes de oscilação serem diferentes, que deverá ser consequência em grande parte
de na análise teórica ter-se considerado o modelo do díodo ideal, sem resistência interna com corrente nula para $V_D < V_{ON}$ e tensão constante $V_D = V_{ON}$ quando ligado, 
pelo que assim não é tido em conta grande parte do comportamento não linear do díodo que é relevante no modelo mais complexo e mais realista utilizado na simulação pelo \emph{Ngspice}.
\\
Pode-se constatar, portanto, que os resultados obtidos na análise teórica e na simulação são compatíveis, apresentando apenas algumas pequenas diferenças no
efeito de \emph{rippling} devidas provavelmente à simplicidade do modelo do díodo utilizado na análise teórica e ainda devido à aproximação linear resultante do método da análise incremental.



\section{Cálculo da Figura de Mérito}
\label{sec:Merit}

A fórmula utilizada para o cálculo do mérito neste trabalho é dada pela expressão (\ref{eq:Merito}).

\begin{equation}
    M = \frac{1}{\textnormal{cost} * (\textnormal{ripple($v_O$)} + \textnormal{average($v_O$ - 12)} + 10^{-6})}
    \label{eq:Merito}
\end{equation}

No circuito desenvolvido neste trabalho utilizaram-se: 22 díodos, 1 condensador ($C = 3.1\: \mu F$) e 1 resistência ($R = 4.9 k\Omega$). 
Assim o custo associado a este circuito (nas unidades monetárias adotadas) é de:

\begin{equation}
  \label{eq:Custo}
  \textnormal{cost} = 22 \times 0.1 + 3.1 \times 1 + 4.9 \times 1 = 10.2 MU 
\end{equation}

Considerando o valor do custo, e o valor médio de $v_O$, $average(v_O) = 12.00071 V$, e da \emph{ripple},
$\textnormal{\emph{ripple}} = 2.245060 \times 10^{-1} \:V$, retirados da tabela (\ref{tab:SimValues})
podemos calcular o mérito deste trabalho obtendo-se:

\begin{equation}
  \label{eq:ValorMerito}
  M = 0.4353100 
\end{equation}

