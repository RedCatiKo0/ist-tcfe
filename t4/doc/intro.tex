\clearpage

\section{Introdução}
\label{sec:introduction}

O objetivo desta atividade laboratorial consiste na concepção de um circuito \emph{Audio Amplifier}, ilustrado na figura
(\ref{fig:Circuito}), que amplifique um sinal sinusoidal de frequência $f = 1 \: kHz$ e amplitude $A = 10 \: mV$, sem
que ocorram distorções visíveis do sinal original, maximizando o \emph{Voltage Gain} e a \emph{Bandwidth}.
\\
Na secção \ref{sec:simulation}, procurou-se simular o \emph{Audio Amplifier}, realizando-se um \emph{Operating Point}
para estudar a parte DC do circuito, uma \emph{Transient Analysis} de $10 \: ms$, de modo a podermos observar o efeito
do amplificador no sinal original e fez-se também uma \emph{AC Analysis} para estudar a \emph{Frequency Response} do
circuito, de onde retirarámos os valores para o \emph{Voltage Gain} e a \emph{Bandwidth}.
Cálculamos ainda as impedâncias de \emph{Input} e \emph{Output} utilizando configurações apropriadas do circuito.
Por outro lado, fizeram-se observações sobre o efeito que alguns dos componentes têm sobre o \emph{Voltage Gain} e 
a \emph{Bandwidth} e realizou-se ainda o cálculo do mérito do circuito de acordo com os critérios definidos para
este trabalho.
\\
O circuito desenvolvido é composto por um \emph{Gain Stage} e um \emph{Output Stage}, cujo funcionamento é analisado
através de modelos teóricos adequados na secção (\ref{sec:analysis}). Para este efeito, realizaram-se os mesmos cálculos
e gráficos feitos na simulação. Nesta secção compararam-se ainda os resultados obtidos através da simulação com os obtidos
via métodos teóricos lado a lado.
\\
Por fim, na conclusão, secção (\ref{sec:conclusion}), resumem-se os resultados principais deste trabalho laboratorial
e recapitulam-se as diferenças entre a simulação e os modelos teóricos encontradas.

\begin{figure}[H] \centering
    \includegraphics[width=0.9\linewidth]{T4Circuit.pdf}
    \caption{Circuito: Amplificador de Som}
    \label{fig:Circuito}
\end{figure}