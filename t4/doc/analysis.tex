\section{Análise Teórica}
\label{sec:analysis}

\subsection{\emph{Operating Point}}

Nesta secção do trabalho, estudamos a parte DC do circuito. Para tal, realizámos um \emph{Operating Point} em
ambos os subcircuitos \emph{Gain Stage} e \emph{Output Stage} separadamente.

\subsubsection{\emph{Gain Stage}}

Começamos por analisar o subcircuito \emph{Gain Stage}, tal como apresentado no esquema elétrico da Figura
(\ref{fig:Circuito}). Para calcular o \emph{Operating Point}, não é necessário considerar os condensadores, já que para
$\omega = 0$ temos $Z_C = 1/(j \omega C) = +\infty$, ou seja, estes comportam-se como um circuito aberto.
\\
Isto permite-nos ainda simplificar o circuito através da utilização de um equivalente de Thévenin.
\\
Tendo em conta estas simplificações podemos obter as seguintes expressões para grandezas de interesse (derivadas e
explicadas na aula teórica 17). Em primeiro lugar, apresentam-se as expressões correspondentes ao equivalente
de Thévenin do \emph{Bias Circuit}:

\begin{equation}
  \label{eq:RB}
  R_B = \frac{R_{B_1} R_{B_2}}{R_{B_1} + R_{B_2}}
\end{equation}

\begin{equation}
  \label{eq:Veq}
  V_{eq} = - \frac{R_{B_2}}{R_{B_1} + R_{B_2}} V_{CC}
\end{equation}

De seguida, apresentam-se as expressões correspondentes às correntes no circuito e às voltagens resultantes da
aplicação de KVL nas malhas e da utilização da lei de Ohm nas resistências e das relações entre as correntes no
transistor NPN:

\begin{equation}
  \label{eq:IE}
  I_E = (1 + \beta_F) I_B
\end{equation}

\begin{equation}
  \label{eq:IB}
  I_B = \frac{V_{eq} - V_{ON}}{R_B + (1 + \beta_F) R_E}
\end{equation}

\begin{equation}
  \label{eq:IC}
  I_C = \beta_F I_B
\end{equation}

\begin{equation}
  \label{eq:VO}
  V_{O_1} = V_{CC} - R_C I_C 
\end{equation}

\begin{equation}
  \label{eq:VE}
  V_E = R_E I_E
\end{equation}

Onde $I_B$ e $I_C$ são as correntes que entram na base e no coletor do transistor, respetivamente, enquanto que $I_E$ é a corrente
que sai do emissor. Nota-se ainda que $V_E$ corresponde à tensão na resistência $R_E$ no sentido de $I_E$ e $V_{O1}$ à diferença de potencial
entre o coletor do transistor e o \emph{ground} (ver figura (\ref{fig:Circuito})).
\\
Podemos calcular ainda a tensão entre o coletor e o emissor do transistor NPN, $V_{CE}$, através da seguinte expressão:

\begin{equation}
  \label{eq:VCE}
  V_{CE} = V_{O_1} - V_E 
\end{equation}

Isto permite verificar se o transistor se encontra a funcionar na região ativa direta (\emph{Forward Active Region}),
já que nesta situação temos $V_{CE} > V_{BE_{ON}}$. Assim, utilizando as expressões anteriores e recorrendo ao
\emph{Octave}, calcularam-se as grandezas características do \emph{Gain Stage} na situação de \emph{Operating Point},
para o mesmo circuito utilizado na simulação. Os valores apresentam-se na tabela (\ref{tab:Teo-OP-GainStage}).

\begin{table}[H]
    \centering
    \begin{tabular}{|l|r|}
    \hline    
    {\bf Quantity} & {\bf Value} \\ \hline
    \input{OP-GainStage}
    \end{tabular}
    \caption{Análise de \emph{Operating Point} - \emph{Gain Stage}}
    \label{tab:Teo-OP-GainStage}
\end{table} 

Assim, podemos confirmar que $V_{CE} > V_{BE_{ON}} = 0.7 \: V$ pelo que o transistor está a funcionar na região adequada
para o funcionamento de um amplificador - o que resulta da utilização do \emph{Bias Circuit}.
\\
Comparando os valores da tabela (\ref{tab:Teo-OP-GainStage}) com os obtidos na simulação, tabelados em (\ref{tab:OPSimValues}),
podemos observar que apesar de os resultados diferirem um pouco, têm a mesma ordem de grandeza o que nos permite estabelecer
as mesmas conclusões para os dois métodos (Transistor NPN na região ativa direta). As diferenças entre os valores devem-se
principalmente do facto de termos utilizado o modelo linear do transistor na análise teórica invês de um modelo mais complexo como é o caso
do utilizado pelo \emph{Ngspice}.


\subsubsection{\emph{Output Stage}}

Tal como anterioremente, nesta secção procura-se determinar um regime de \emph{Operating Point}, mas para o \emph{Output
Stage}. Temos novamente em conta que no \emph{Operating Point} os condensadores correspondem a circuitos abertos.
Assim da análise do circuito resultam as seguintes expressões:

\begin{equation}
  I_{out} = \frac{V_{CC} - V_{EB_{ON}} - V_{O_1}}{R_{out}}
\end{equation}

\begin{equation}
  V_{EC} = \frac{V_{CC} - V_{EB_{ON}} - V_{O_1}}{R_{out}}
\end{equation}

\begin{equation}
  I_{col_2} = \frac{\beta_F}{\beta_F+1}  I_{out}
\end{equation}

Onde consideramos $V_{EB_{ON}} = 0.7 \: V$ e que $I_{col_2}$ corresponde à corrente que sai do coletor do transistor da \emph{output stage}
e $I_{out}$ à corrente na resistência $R_{out}$.
Calculando o \emph{Operating Point} do \emph{Output Stage}, obtemos os valores da tabela (\ref{tab:Teo-OP-OutputStage}).

\begin{table}[H]
    \centering
    \begin{tabular}{|l|r|}
    \hline
    {\bf Quantity} & {\bf Value} \\ \hline
    \input{OP-OutputStage}
    \end{tabular}
    \caption{Análise de \emph{Operating Point} - \emph{Output Stage}}
    \label{tab:Teo-OP-OutputStage}
\end{table}

Podemos notar que $V_{EC} > V_{EB_{ON}}$ pelo que o transistor PNP está a funcionar na região adequada para o seu funcionamento como 
amplificador.
\\
Por outro lado comparando os valores da tabela (\ref{tab:Teo-OP-OutputStage}) com os obtidos na simulação, tabelados em (\ref{tab:OPSimValues}),
podemos notar novamente que os valores obtidos pela análise teórica têm a mesma ordem de grandeza dos obtidos pela simulação, apenas
possuindo pequenos desvios entre si. Tal como anteriormente, as diferença entre os modelos devem-se ao uso de um modelo linear para
representar o transistor PNP.


\subsection{Ganho e Impedâncias Totais e de cada \emph{stage}}

Nesta secção do trabalho começa-se por realizar a análise incremental de cada um dos subcircuitos, calculando-se assim o seu ganho.
De seguida, determinam-se as impedâncias de \emph{Input} e \emph{Output} de cada um dos andares, o que permitirá por fim calcular a
impedância de \emph{Input} e de \emph{Output} do circuito.

\subsubsection{Análise Incremental - \emph{Gain Stage}}

Começamos por analisar o circuito sem o efeito do condensador de \emph{bypass}.
\\
Os parâmetros do modelo incremental podem ser calculados recorrendo aos valores determinados na análise em \emph{operating point}.
Estes são dados pelas seguintes relações:

\begin{equation}
  g_{m} = \frac{I_C}{V_T}
\end{equation}

\begin{equation}
  r_{\pi} = \frac{\beta_F}{g_m}
\end{equation}

\begin{equation}
  r_o \approx \frac{V_A}{I_C}
\end{equation}

Onde nas relações anteriores $V_A = 69.7 V$ e $\beta_F = 178.7$ são propriedades do transistor npn utilizado.
Resolvendo o circuito incremental podemos obter o ganho associado ao \emph{gain stage} que é dado por:

\begin{equation}
  \frac{v_{o1}}{v_{s}} = \frac{R_B R_C}{R_S + R_B} \frac{R_E - g_m r_{\pi} r_o}{(r_o + R_C + R_E)(R_B || R_S + r_{\pi} + R_E) + g_m R_E r_o r_{\pi} - R_E^2}
\end{equation}

Incluindo agora o condensador de \emph{bypass} no circuito anterior, sabemos que para frequências médias/altas 
podemos considerar que o condensador age como um fio ($Z_C = 1/(j \omega C) \approx 0$ para $\omega$ elevados). Isto permite contrariar a 
diminuição do ganho devida à resistência $R_E$. Assim a expressão anterior reduz-se a:

\begin{equation}
  \frac{v_{o1}}{v_{s}} = \frac{R_B R_C}{R_S + R_B} \frac{- g_m r_{\pi} r_o}{(r_o + R_C)(R_B || R_S + r_{\pi})}
\end{equation}

Por fim, recorrendo ao \emph{Octave} determinaram-se os valores em causa recorrendo às expressões anteriores. Estes encontram-se na tabela (\ref{tab:Teo-AC-Gain-Stage}).

\begin{table}[H]
    \centering
    \begin{tabular}{|l|r|}
    \hline    
    {\bf Quantity} & {\bf Value} \\ \hline
    \input{AC-GainStage}
    \end{tabular}
    \caption{Análise Incremental - \emph{Gain Stage}}
    \label{tab:Teo-AC-Gain-Stage}
\end{table}

Podemos verificar que o o ganho quando se utiliza condensador de \emph{bypass} é muito superior ao ganho sem condensador como seria de esperar.

\subsubsection{Análise Incremental - \emph{Output Stage}}

Novamente recorremos ao modelo incremental do circuito onde os parâmetros correspondentes podem ser calculados 
recorrendo aos valores determinados na análise em \emph{operating point}.
Neste caso trabalhamos com as admitâncias por coveniência:

\begin{equation}
  g_{m} = \frac{I_{col2}}{V_A}
\end{equation}

\begin{equation}
  g_{\pi} = \frac{1}{r_{\pi}}
\end{equation}

\begin{equation}
  g_{E} = \frac{1}{R_{E}}
\end{equation}

\begin{equation}
  g_{o} = \frac{1}{r_{o}}
\end{equation}

Onde nas relações anteriores $V_A = 37.2 V$ e $\beta_F = 227.3$ são propriedades do transistor pnp utilizado.
Resolvendo o circuito incremental podemos obter o ganho associado ao \emph{gain stage} que é dado por:

\begin{equation}
  \frac{v_{o_2}}{v_{s}} = \frac{g_m}{g_m + g_{\pi} + g_E + g_o}
\end{equation}

Por fim, recorrendo ao \emph{Octave} determinaram-se os valores em causa recorrendo às expressões anteriores. Estes encontram-se na tabela (\ref{tab:AC-OutputStage}).

\begin{table}[H]
    \centering
    \begin{tabular}{|l|r|}
    \hline    
    {\bf Quantity} & {\bf Value} \\ \hline
    \input{AC-OutputStage}
    \end{tabular}
    \caption{Análise Incremental - \emph{Output Stage}}
    \label{tab:AC-OutputStage}
\end{table}

\subsubsection{Impedâncias e Ganho Total}

De seguida calculam-se as impedâncias de \emph{input} e \emph{output} associadas ao circuito onde se utilizaram sobrescritos 1 e 2
para identificar grandezas relativas ao \emph{gain stage} e ao \emph{output stage} respetivamente. \\
As impedâncias correspondentes à \emph{gain stage} para o caso em que $R_E = 0$ (consideramos o condensador de \emph{bypass} e
frequências elevadas) são dadas por:

\begin{equation}
  Z_{I1} = R_B || r_{\pi} = \frac{R_B r_{\pi}}{R_B + r_{\pi}}
\end{equation}

\begin{equation}
  Z_{O1} = R_C || r_{o} = \frac{R_C r_{o}}{R_B + r_{o}}
\end{equation}

Por outro lado as impedâncias de \emph{input} e \emph{output} correspondentes à \emph{output stage} são dadas por:

\begin{equation}
  Z_{I2} = \frac{g_{\pi_2} + g_{E_2} + g_{o_2} + g_{m_2}}{g_{\pi_2}(g_{\pi_2} + g_{E_2} + g_{o_2})}
\end{equation}

\begin{equation}
  Z_{O2} = \frac{1}{g_{\pi_2} + g_{E_2} + g_{o_2} + g_{m_2}}
\end{equation}

Podemos assim relacionar as impedâncias de \emph{imput} e \emph{output} do circuito, $Z_I$ e $Z_O$, com as impedâncias anteriores
da seguinte forma:

\begin{equation}
  Z_I = Z_{I_1}
\end{equation}

\begin{equation}
  Z_O = \frac{1}{g_{o_2} + g_{m_2} + \frac{r_{\pi_2}}{r_{\pi_2} + Z_{o_1}} + g_{E_2} + \frac{1}{r_{\pi_2} + Z_{O_1}}}
\end{equation}

Aproveitamos ainda para calcular o ganho total associado ao circuito:

\begin{equation}
  \frac{v_{out}}{v_{s}} = \frac{\frac{1}{r_{\pi_2} + Z_{o_1}} + \frac{g_{m_2} r_{\pi_2}}{r_{\pi_2} + Z_{o_1}}}{\frac{1}{r_{\pi_2} + Z_{o_1}} + \frac{1}{R_{out}} + \frac{1}{r_{o_2} + \frac{g_{m_2} r_{\pi_2}}{r_{\pi_2} + Z_{o_1}}} } A_1
\end{equation}

Onde $A_1$ é o ganho associado à \emph{gain stage}. \\
Por fim apresentam-se os valores destas grandezas para o circuito simulado na tabela (\ref{tab:Teo-Impedancias}).

\begin{table}[H]
    \centering
    \begin{tabular}{|l|r|}
    \hline    
    {\bf Quantity} & {\bf Value} \\ \hline
    \input{Impedances}
    \end{tabular}
    \caption{Ganho Total e Impedâncias do \emph{Gain Stage}, do \emph{Output Stage} e Totais}
    \label{tab:Teo-Impedancias}
\end{table}

Verifica-se que a impedância de \emph{input} do circuito é elevada e que a impedância de \emph{output} do circuito é da mesma ordem de grandeza
que a impedância da \emph{load} o que corresponde à situação ideal.
\\
Podemos também notar que a \emph{gain stage} e a \emph{output stage} podem ser ligadas sem perda significativa de sinal dado que a impedância
de \emph{output} da \emph{gain stage} é muito menor do que a impedância de \emph{input} da \emph{outputstage} pelo que o sinal 
do primeiro andar é transmitido quase na totalidade ao segundo andar.
\\
Por último podemos verificar que as grandezas da tabela (\ref{tab:Teo-Impedancias}) que foram calculadas pela via teórica
estão na mesma ordem de grandeza que os valores obtidos através da simulação, ver tabela (\ref{tab:ACSimValues}), sendo que como já foi
mencionado anteriormente, a diferença entre ambas se deve dever à não lineariedade dos transistores (o \emph{Ngspice} utiliza um modelo mais
realista do transistor) e à aproximação dos condensadores a fios (que apesar de tudo deverá ser bastante precisa para médias/elevadas frequências).

\subsection{\emph{Frequency Response}}

Nesta secção faz-se uma análise no domínio das frequências do ganho da voltagem de \emph{output} da \emph{gain stage} e do circuito completo.
\\
Para tal, resolveu-se o modelo incremental do circuito, ilustrado na Figura (\ref{fig:FrequencyResponseCircuitoIncremental}) utilizando o método dos nós.

\begin{figure}[H]
    \centering
    \includegraphics[scale=0.38]{T4IncCircuit.pdf}
    \caption{\emph{Frequency Response} - Circuito Incremental}
    \label{fig:FrequencyResponseCircuitoIncremental}
\end{figure}

O sistema de equações obtidos em forma matricial é dado por:

\begin{equation}
\resizebox{.9\hsize}{!}
{$
\begin{bmatrix}
    1 & 0 & 0 & 0 & 0 & 0 & 0\\
    -\frac{1}{R_S} & \frac{1}{R_S} + j \omega C_I & - j \omega C_I & 0 & 0 & 0 & 0\\
    0 & - j \omega C_I & j \omega C_I + \frac{1}{R_{B}} + \frac{1}{r_{\pi_1}} & - \frac{1}{r_{\pi_1}} & 0 & 0 & 0\\
    0 & 0 & -\frac{1}{r_{\pi_1}} - g_{m_1} & \frac{R_E + r_{\pi_1}}{r_{\pi_1} R_E} + \frac{1}{r_{o1}} + g_{m_1} + j \omega C_E & - \frac{1}{r_{o_1}} & 0 & 0\\
    0 & 0 & g_{m_1} & - g_{m_1} - \frac{1}{r_{o_1}} & \frac{1}{R_C} + \frac{1}{r_{o_1}} + \frac{1}{r_{\pi_2}} & -\frac{1}{r_{\pi_2}} & 0\\
    0 & 0 & 0 & g_{m_2} & -g_{m_1} - \frac{1}{r_{\pi}} & \frac{1}{r_{o_2}} + \frac{1}{r_{\pi_2}} + \frac{1}{R_{out}} + j \omega C_{out}  & - j \omega C_{out} \\ 
    0 & 0 & 0 & 0 & 0 & -j \omega C_{out} & j \omega C_{out} + \frac{1}{R_2} \\
\end{bmatrix}
\begin{bmatrix}
  \widetilde{v_1}\\
  \widetilde{v_2}\\
  \widetilde{v_3}\\
  \widetilde{v_5}\\
  \widetilde{v_6}\\
  \widetilde{v_7}
\end{bmatrix}
=
\begin{bmatrix}
  \widetilde{v_s}\\
  0\\
  0\\
  0\\
  0\\
  0\\
  0
\end{bmatrix}
$}
\end{equation}

Onde $\widetilde{v_S} = 10 e^{-\frac{\pi}{2}} \: mV$. Assim resolvendo o sistema matricial anterior de forma simbólica no \emph{Octave}
obtiveram-se as voltagens nodais como funções da frequência, $f$, em Hz. Isto permitiu fazer \emph{plot's} da magnitude e da fase 
da função de transferência da voltagem de saída do \emph{gain stage} ($T_1(\omega) = \frac{\widetilde{V}_{o_1}}{\widetilde{V}_{S}} = \frac{\widetilde{V}_{5}}{\widetilde{V}_{S}} $)
e da voltagem nos terminais da \emph{load} ($T_1(\omega) = \frac{\widetilde{V}_{out}}{\widetilde{V}_{S}} = \frac{\widetilde{V}_{7}}{\widetilde{V}_{S}}$).
\\
Os gráficos em causa apresentam-se de seguida onde as amplitudes estão representadas em dB e as fases em graus.

\begin{figure}[H]
    \centering
    \includegraphics[scale=0.38]{FrequencyResponseMagGainStage.eps}
    \includegraphics[scale=0.38]{FrequencyResponsePhaseGainStage.eps}
    \caption{\emph{Frequency Response} - Magnitude (dB) e Fase (º) da Função de Transferência do Output da \emph{Gain Stage} - $T_1(\omega)$}
    \label{fig:FrequencyResponseAnaGainStage}
\end{figure}

\begin{figure}[H]
    \centering
    \includegraphics[scale=0.38]{FrequencyResponseMagOutput.eps}
    \includegraphics[scale=0.38]{FrequencyResponsePhaseOutput.eps}
    \caption{\emph{Frequency Response} - Magnitude (dB) e Fase (º) da Função de Transferência do Output na \emph{Load} - $T_2(\omega)$}
    \label{fig:FrequencyResponseAnaOutput}
\end{figure}

Os gráficos anteriores são semelhantes em alguns aspetos aos produzidos pela simulação (ver Figuras (\ref{fig:ACASimMag}) e (\ref{fig:ACASimPha})).
Nos gráficos de magnitude as ordens de grandeza do ganho são bastante semelhantes e o comportamento do gráfico é semelhante até cerca de $f \approx 10^6 Hz$
onde deverá existir um polo na função de transferência que causa uma diminuição do ganho. 
\\
Os gráficos de fase apresentam comportamentos muito distintos não sendo comparáveis.
\\
Estas diferenças dever-se-ão fundamentalmente à simplicidade do modelo teórico utilizado na descrição do transistor nomeadamente dado que 
não parece existir nenhum polo no modelo teórico utilizado que justificasse a diminuição do ganho para altas frequências.
\\
Esta análise permite concluir que apesar do modelo teórico simplificado oferecer boas aproximações para o estudo de circuitos com transistores
poderá ser necessário recorrer a simulações numéricas quando for desejável efetuar um estudo mais cuidado.