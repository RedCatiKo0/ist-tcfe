\section{Conclusão}
\label{sec:conclusion}

Neste trabalho desenvolvemos um Amplificador de Audio, utilizando um \emph{Gain Stage} e um \emph{Output Stage} que, como verificámos
nas secções de análise teórica (\ref{sec:analysis}) e através da simulação (\ref{sec:simulation}), produzem resultados satisfatórios.
\\
O estudo do circuito resultou no cálculo de várias grandezas (\emph{Voltage Gain} e \emph{Bandwidth}) e \emph{plot}'s de gráficos que
foram obtidos através da simulação e da análise teórica. Estes resultados foram apresentados e discutidos nas respetivas secções e
comparados na secção da análise teórica (\ref{sec:analysis}), tendo-se verificado algumas diferenças entre o modelo teórico e a simulação
do \emph{Ngspice}, no entanto os resultados são da mesma ordem de grandeza pelo que permitem derivar as mesmas conclusões.
Estas diferenças deverão ser consequência da lineariedade dos modelos utilizados na secção teórica (modelo linear do transistor 
e análise incremental) que, apesar de apresentarem resultados satisfatórios (consistentes com os da simulação), 
não são capazes de descrever a vertente não linear do circuito com a mesma precisão que o modelo mais complexo utilizado pelo \emph{Ngpsice}.
\\
O mérito do trabalho foi calculado na secção (\ref{sec:simulation}), através da determinação do custo dos componentes utilizados no circuito
e com os valores do \emph{Voltage Gain}, da \emph{Bandwidth} e da \emph{Lower Cut Off Frequency}, determinados através da simulação.
\\
Concluindo, consideramos que os objetivos do trabalho foram atingidos tendo sido possível desenvolver e estudar um Amplificador de Audio
tanto através de métodos teóricos como através de simulações numéricas.