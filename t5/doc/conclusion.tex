\section{Conclusão}
\label{sec:conclusion}

Neste trabalho desenvolvemos um \emph{bandpass filter} com um ganho de $40 \: dB$ na frequência central de $1 \: kHz$ que, como
verificámos nas secções de análise teórica (\ref{sec:analysis}) e através da simulação (\ref{sec:simulation}), produzem resultados
muito satisfatórios.
\\
O estudo do circuito resultou no cálculo de várias grandezas tais como o ganho, a frequência central e as impedâncias de \emph{input}
e \emph{output} e \emph{plot}'s de gráficos que foram obtidos através da simulação e da análise teórica.
Estes resultados foram apresentados e discutidos nas respetivas secções e comparados na secção da análise teórica (\ref{sec:analysis}),
tendo-se verificado que os resultados obtidos no modelo teórico e na simulação do \emph{Ngspice} são muito semelhantes. No entanto,
existem algumas diferenças pouco significativas que são consequências da lineariedade dos modelos utilizados na secção teórica - modelo
ideal do \emph{OpAmp} que, apesar de causar pequenos desvios entre os valores permite descrever o circuito com uma elevada precisão, tendo ainda
a vantagem de ser mais simples do que o modelo utilizado pelo \emph{Ngspice}.
\\
O mérito do trabalho foi calculado na secção (\ref{sec:simulation}), através da determinação do custo dos componentes utilizados no
circuito e com os valores do ganho e da frequência central, determinados através da simulação.
\\
Concluindo, consideramos que os objetivos do trabalho foram atingidos tendo sido possível desenvolver e estudar um \emph{bandpass
filter} tanto através de métodos teóricos como através de simulações numéricas.