\clearpage

\section{Introdução}
\label{sec:introduction}

O objetivo desta atividade laboratorial consiste na concepção de um circuito \emph{bandpass filter} utilizando 
um \emph{Operational Amplifier}, ou \emph{OP-AMP}, ilustrado na figura (\ref{fig:Circuito}), que amplifique uma banda
de frequências centrada em $f_c = 1 \: kHz$, com um ganho de $gain = 40 \: dB$.
\\
Na secção \ref{sec:simulation}, procurou-se simular o \emph{bandpass filter} utilizando o modelo não ideal do
\emph{OP-AMP} fornecido, efetuando-se uma \emph{AC analysis} para estudar a \emph{frequency response} do circuito,
de onde retirarámos os valores para a frequência central e o respetivo ganho nessa frequência.
Calcularam-se ainda as impedâncias de \emph{input} e \emph{output}, utilizando configurações apropriadas do circuito.
Finalmente, realizou-se ainda o cálculo do mérito do circuito de acordo com os critérios definidos para este trabalho.
\\
O funcionamento do circuito desenvolvido foi analisado através de modelos teóricos adequados na secção (\ref{sec:analysis}).
Para este efeito, realizaram-se os mesmos cálculos e gráficos feitos na simulação. Nesta secção compararam-se ainda os
resultados obtidos através da simulação com os obtidos via métodos teóricos lado a lado.
\\
Por fim, na conclusão, secção (\ref{sec:conclusion}), resumem-se os resultados principais deste trabalho laboratorial
e recapitulam-se as diferenças entre a simulação e os modelos teóricos encontradas.

\begin{figure}[H] \centering
    \includegraphics[width=0.9\linewidth]{t5circuito.pdf}
    \caption{Circuito: \emph{Bandpass filter}}
    \label{fig:Circuito}
\end{figure}