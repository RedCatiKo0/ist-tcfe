\clearpage

\section{Introduction}
\label{sec:introduction}

In this laboratory assignment, we study the circuit ilustrated in the figure (\ref{fig:1}).
Our objective is to better understand how an RC circuit works and how it responds to certain stimuli.
More specifically, we studied the circuit in its stationary state ($t < 0 \: s$), as well as how it
responded to a sine wave with frequency $f$: $v_s(t) = \sin(2 \pi f t)$, when $t > 0 \: s$.
\\
In section (\ref{sec:analysis}), we performed a theoretical analysis of the circuit.
Using the Nodal Method, we described its behavior when $t < 0 \: s$, computed the equivalent
resistor as seen by the capacitor to obtain the natural solution after $t > 0 \: s$,
and through the computation of phasor voltages we obtained the forced solution and studied
the frequency response.
\\
In section (\ref{sec:simulation}), we present the results derived from various simulations of the circuit
that were obtained from \emph{Ngspice} scripts. We simulated the linear circuit when $t < 0 \: s$, the total solution
when the circuit is subjected to the stimulus and its frequency response.
Moreover, we also compare the values determined through theoretical methods and the values derived from
the circuit simulations side by side, explaining why they did or did not differ from one another.
\\
Finally, in section (\ref{sec:conclusion}), we sumarize the results of this assignment, namely the
comparison between the theoretical analysis and the simulation.

\begin{figure}[H] \centering
    \includegraphics[width=0.7\linewidth]{t2cnormal.pdf}
    \caption{RC Circuit}
    \label{fig:1}
\end{figure}