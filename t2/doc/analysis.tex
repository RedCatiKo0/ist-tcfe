\section{Theoretical Analysis}
\label{sec:analysis}

In this section we analyse the circuit shown in Figure (\ref{fig:1}) before and 
after $t = 0 \:s$ using the theoretical models studied in class.

\subsection{Nodal Analysis Method - $t < 0 \:s$}
\label{se:2.1}

We start by analysing the stationary regime of the circuit, or in 
other words, we assume enough time has passed
so that the capacitor is fully charged.
The capacitor is described by 
equation (\ref{eq:CapacitorLaw}).

\begin{equation}
  i_c = C \frac{d v_c}{dt}
  \label{eq:CapacitorLaw}
\end{equation}

Where $i_c$ and $v_c$ are, respectively, the current passing through the capacitor and the voltage at
it's terminals. Therefore if we are in a stationary situation, that is $v_c = constant$, then $i_c = 0 \: A$. 
The capacitor has an open circuit behavior which implies the circuit is equivalent to the one 
in figure (\ref{fig:2}).

\begin{figure}[H] \centering
  \includegraphics[width=0.7\linewidth]{t2csemcondensador.pdf}
  \caption{RC Circuit to be analysed with Node Voltages conventions}
  \label{fig:2}
\end{figure}

Using the nodal voltage analysis method we can obtain 7 linear equations for $V_1$, $V_2$, $V_3$, $V_5$, $V_6$, $V_7$ and $V_8$,
which we now present in matricial form.

\begin{equation}
  \begin{bmatrix}
    1 & 0 & 0 & 0 & 0 & 0 & 0\\
    -\frac{1}{R_1} & \frac{1}{R_1} + \frac{1}{R_2} + \frac{1}{R_3} & - \frac{1}{R_2} & - \frac{1}{R_3} & 0 & 0 & 0\\
    0 & \frac{1}{R_2} + K_b & - \frac{1}{R_2} & - K_b & 0 & 0 & 0\\
    0 & - \frac{1}{R_3} & 0 & \frac{1}{R_3} + \frac{1}{R_4} + \frac{1}{R_5} & - \frac{1}{R_5} & - \frac{1}{R_7} & \frac{1}{R_7}\\
    0 & K_b & 0 & - \frac{1}{R_5} - K_b & \frac{1}{R_5} & 0 & 0\\
    0 & 0 & 0 & 0 & 0 & \frac{1}{R_6} + \frac{1}{R_7}  & - \frac{1}{R_7} &\\ 
    0 & 0 & 0 & 1 & 0 & \frac{K_d}{R_6} & -1\\
  \end{bmatrix}
  \begin{bmatrix}
    V_1\\
    V_2\\
    V_3\\
    V_5\\
    V_6\\
    V_7\\
    V_8
  \end{bmatrix}
  =
  \begin{bmatrix}
    V_s\\
    0\\
    0\\
    0\\
    0\\
    0\\
    0
  \end{bmatrix}
\end{equation}

By solving this equations using \emph{Octave} we obtained the values in table (\ref{tab:Teo1}).
In this table we also present the values for the currents in all branches, $I_{Rn}$, in the direction
defined as from the lower to the higher numbered node, in order to match the ones calculated in 
\emph{Ngspice}, as well as $V_b$ and $V_d$.

\begin{table}[H]
    \centering
    \begin{tabular}{|l|r|}
    \hline    
    {\bf Quantity} & {\bf Value} \\ \hline
    \input{Teo1-NodalMethod}
    \end{tabular}
    \caption{Nodal Analysis Results - $t < 0 \:s$}
    \label{tab:Teo1}
\end{table} 



\subsection{Equivalent Resistance, $R_{eq}$}
\label{sec:AnalEqRes}

In this section we want do determine the equivalent resistance, $R_{eq}$, of the circuit
as seen from the capacitor terminals. To do this we consider the Thévenin equivalent of
the circuit as seen in figure (\ref{fig:3}). 

\begin{figure}[H] \centering
  \includegraphics[width=0.7\linewidth]{t2thev.pdf}
  \caption{Thévenin Equivalent Circuit}
  \label{fig:3}
\end{figure}

To calculate the equivalent resistance we turn off all independent sources in the circuit. This 
corresponds to replacing the voltage source in the Thévenin equivalent with a short circuit.
Therefore, by replacing the capacitor with a voltage source that imposes a voltage $V_x = V_6 - V_8$, where
$V_6$ and $V_8$ are the nodal voltages determined in the previous section we have a simple way to calculate $R_{eq}$.
We only need to determine $I_x$, that is, the current that is supplied by the voltage source $V_x$ to the equivalent
resistance. By doing this, we can use Ohm's Law to determine $R_{eq} = \frac{V_x}{I_x}$.
\\
All this considered, to calculate $I_x$ we use nodal analysis in the circuit presented in figure (\ref{fig:4}).

\begin{figure}[H] \centering
  \includegraphics[width=0.7\linewidth]{t2csemfontetensao.pdf}
  \caption{Circuit to be analysed with Node Voltages conventions}
  \label{fig:4}
\end{figure} 

This whole procedure is necessary due to the fact that the circuit in question has several resistances and two dependent sources which complicates
the calculation of $R_{eq}$. Furthermore it is useful to use $V_x = V_6 - V_8$ because by doing so, we will not only determine the equivalent resistance, $R_{eq}$,
but also calculate the nodal voltages of all nodes in instant $t = 0^{+} \: s$. This is a result of the fact that
$v_s(0^{+}) = 0\: V$ (at that instant the voltage source is equivalent to a wire) and that the voltage of the capacitor is continuous at $t = 0$ which implies that $v_6(0^{+}) - v_8(0^{+}) = V_x$.
\\
The equations correspondent to the nodal analysis are presented in matricial form bellow:

\begin{equation}
  \begin{bmatrix}
    \frac{1}{R_2} + \frac{1}{R_3} + \frac{1}{R_4} & - \frac{1}{R_2} & - \frac{1}{R_3} & 0 & 0 & 0\\
    \frac{1}{R_2} + K_b & - \frac{1}{R_2} & - K_b & 0 & 0 & 0\\
    0 & 0 & 1 & 0 & \frac{K_d}{R_6} & -1\\
    0 & 0 & 0 & 1 & 0 & - 1\\
    0 & 0 & 0 & 0 & \frac{1}{R_6} + \frac{1}{R_7} & - \frac{1}{R_7}\\
    -\frac{1}{R_3} + K_b & 0 & \frac{1}{R_3} + \frac{1}{R_4} - K_b & 0 & -\frac{1}{R_7} & \frac{1}{R_7} \\ 
  \end{bmatrix}
  \begin{bmatrix}
    V_2\\
    V_3\\
    V_5\\
    V_6\\
    V_7\\
    V_8

  \end{bmatrix}
  =
  \begin{bmatrix}
    0\\
    0\\
    0\\
    V_x\\
    0\\
    0
  \end{bmatrix}
\end{equation}

The equivalent current can be calculated using expression (\ref{eq:Ix}).

\begin{equation}
  I_x = \frac{V_6 - V_5}{R_5} + I_b
  \label{eq:Ix}
\end{equation}

We can also calculate the time constant associated to the circuit:

\begin{equation}
  \tau = R_{eq} C
  \label{eq:TimeConstant}
\end{equation}

Where C is the capacitor's capacity.
Solving the system of equations above and by using expressions (\ref{eq:Ix}) and (\ref{eq:TimeConstant})
we obtained the values in table (\ref{tab:Teo2}).

\begin{table}[H]
    \centering
    \begin{tabular}{|l|r|}
    \hline    
    {\bf Quantity} & {\bf Value} \\ \hline
    \input{Teo2-EquivalentResistance}
    \end{tabular}
    \caption{Nodal Analysis Results - $t = 0^{+} \:s$}
    \label{tab:Teo2}
\end{table}


\subsection{Natural Solution, $v_{6n}(t)$ when $t > 0 \:s$}

In this section we determine the natural solution for the voltage at node 6 after $t > 0 \:s$.
From the theory classes we know the general formula for the voltage of an RC circuit. 
Therefore considering the circuit in figure (\ref{fig:1}) we know that the natural solution for the voltage in node 6 
will be given by an expression of the form:

\begin{equation}
  v_{6n} = v_{6n}(0) e^{-\frac{t}{\tau}}
  \label{eq:NaturalSolutionExpression}
\end{equation}

Considering the results obtained in the previous section we know that $v_{6}(0^{+}) - v_{8}(0^{+}) = V_x$ and that $v_{8}(0^{+}) = 0 \:V$.
\\
Therefore the natural solution for the voltage in node 6 is given by:

\begin{equation}
  v_{6n} = 8.824136 e^{-\frac{t}{3.146672 \times 10^{-3}}}
  \label{eq:NaturalSolution}
\end{equation}

The solution is plotted in figure (\ref{fig:NaturalSolution}) for the interval $[0, 20] \:ms$.

\begin{figure}[H] \centering
\includegraphics[width=0.8\linewidth]{NaturalSolution.eps}
\caption{Natural Solution of the Voltage at Node 6, $v_{6n}(t)$}
\label{fig:NaturalSolution}
\end{figure}


\subsection{Forced Solution, $v_{6f}(t)$ when $t > 0 \:s$}

In this section we determine the forced solution for the voltage at node 6 with $t > 0 \: s$
To do so we consider once again the circuit shown in figure (\ref{fig:1}).

Using node analysis with phasors we can determine the node amplitude and phase for each one of the nodes in the circuit
considering only the forced regime.
Taking into acount Ohm's Law using impedances and that $Z_C = \frac{1}{j \omega C}$ for capacitors and $Z_R = R$ 
for resistances we obtain the following equations in matricial form:

\begin{equation}
  \begin{bmatrix}
    1 & 0 & 0 & 0 & 0 & 0 & 0\\
    -\frac{1}{R_1} & \frac{1}{R_1} + \frac{1}{R_2} + \frac{1}{R_3} & - \frac{1}{R_2} & -\frac{1}{R_3} & 0 & 0 & 0\\
    0 & \frac{1}{R_2} + K_b & -\frac{1}{R_2} & -K_b & 0 & 0 & 0\\
    0 & 0 & 0 & 1 & 0 & \frac{K_d}{R_6} & - 1\\
    0 & K_b & 0 & -K_b - \frac{1}{R_5} & \frac{1}{R_5} + j \omega C & 0 & - j \omega C\\
    0 & 0 & 0 & 0 & 0 & \frac{1}{R_6} + \frac{1}{R_7} & -\frac{1}{R_7} \\ 
    0 & -\frac{1}{R_3} & 0 & \frac{1}{R_3} + \frac{1}{R_4} + \frac{1}{R_5} & - j \omega C - \frac{1}{R_5} & -\frac{1}{R_7} & \frac{1}{R_7} + j \omega C \\
  \end{bmatrix}
  \begin{bmatrix}
    \widetilde{V_1}\\
    \widetilde{V_2}\\
    \widetilde{V_3}\\
    \widetilde{V_5}\\
    \widetilde{V_6}\\
    \widetilde{V_7}\\
    \widetilde{V_8}

  \end{bmatrix}
  =
  \begin{bmatrix}
    \widetilde{V_s}\\
    0\\
    0\\
    0\\
    0\\
    0\\
    0
  \end{bmatrix}
\end{equation}

Where $\widetilde{V_s} = - j$ is the phasor of the voltage source $v_s$.
\\
Using \emph{Octave} we can solve the system above and obtain the corresponding amplitudes for the nodal phasors which
are presented in table (\ref{tab:Teo3}).

\begin{table}[H]
    \centering
    \begin{tabular}{|l|r|}
    \hline    
    {\bf Quantity} & {\bf Value} \\ \hline
    \input{Teo3-ForcedSolution}
    \end{tabular}
    \caption{Forced Solution - Magnitude of Voltage Node Phasors}
    \label{tab:Teo3}
\end{table} 

Considering $\omega = 2 \pi f$ with a linear frequency of $f = 1 \:kHz$ we can determine the forced solution on node 6, $v_{6f}$,
by taking the real part of $\widetilde{V_6} \cdot e^{\omega t}$. In figure (\ref{fig:ForcedSolution}) we plot the forced solution in the interval $[0, 20] \: ms$.

\begin{figure}[H] \centering
\includegraphics[width=0.8\linewidth]{ForcedSolution.eps}
\caption{Forced Solution of the Voltage at Node 6, $v_{6f}(t)$}
\label{fig:ForcedSolution}
\end{figure}

\subsection{Total Solution}

In this section we determine the total solution of the voltage at node 6 for values $t > 0 \:s$
by superimposing the natural and forced solutions calculated previously. 
To compare the results with the behavior of the voltage source $v_s$ 
we plot both $v_s(t)$ and $v_6(t)$ in the interval $[-5, 20] \:ms$.
The values of $v_6(t)$ when $t < 0 \: s$ correspond to the ones calculated in section (\ref{se:2.1}). 
The plot in question is presented in figure (\ref{fig:TotalSolution}).
\\
We can see that the difference between the phases of the two graphs is approximately $\pi$, this result will be confirmed in the next section.

\begin{figure}[H] \centering
\includegraphics[width=0.8\linewidth]{TotalSolution.eps}
\caption{\textcolor{green}{Green - Total Solution of the Voltage at Node 6 ($v_6(t)$)} and \textcolor{red}{Red - Voltage Source ($v_s(t)$)} plots with $t \in [-5, 20] \: ms$}
\label{fig:TotalSolution}
\end{figure}



\subsection{Frequency Responses}
\label{sec:AnalFRes}

In this section we analyse the frequency responses of $v_c(f) = v_6(f) - v_8(f)$, $v_6(f)$ and $v_s(f)$ for the 
frequency range $f \in [0.1, 10^6] \:Hz$.
\\
Using \emph{Octave} we solved symbolically the nodal method analysis equations for the forced regime.
By doing so we obtained expressions for each nodal voltage phasor as a function of $f$. From there we just divided 
the phasor in question ($\widetilde{V_c}$, $\widetilde{V_6}$ or $\widetilde{V_s}$) by $\widetilde{V_s}$ to obtain
the transformation function. 
\\
The absolute value of the transformation functions in decibels and their phase's in degrees
are plotted in graph (\ref{fig:FrequencyResponseAna}).
\\
When analysing the magnitudes we can see that the transformation function yields a magnitude of 1 ($0\: dB$) for $v_s$, this is obviously
a trivial case. On the other hand, we can see that the transformation function regarding the voltage of the capacitator goes to 0 ($-\infty\: dB$)
as the frequency goes to infinity. This means that the capacitor only let's through lower frequencies which is 
coherent with what we have seen in theory classes.
\\
As the frequency tends to infinity the capacitor starts to have the behaviour of a short-circuit therefore the capacitor acts in a similar fashion to a regular
wire which explains why the magnitude of $v_6(f)$ assumes a stationary value.

\begin{figure}[H]
    \centering
    \includegraphics[scale=0.38]{FrequencyResponseMag.eps}
    \includegraphics[scale=0.38]{FrequencyResponsePhase.eps}
    \caption{Frequency Response - Magnitude and Phase: \textcolor{red}{Red - Voltage Source}; \textcolor{green}{Green - Node 6}; \textcolor{blue}{Blue - Capacitor}}
    \label{fig:FrequencyResponseAna}
\end{figure}