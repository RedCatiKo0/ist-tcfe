\section{Simulation Analysis}
\label{sec:simulation}

In this section, we simulate the RC circuit in its different stages in time using \emph{Ngspice},
performing Operating Point Analysis to get the branch currents and node voltages
in the first two subsections. We also perform a Transient Analysis to get the voltage
in node 6 and compare it with the stimulus in node 1 in the third and fourth subsections.
Finally we do an AC Frequency Responce Analysis, in order to compare the Magnitudes
and Phases of the input source, node 6 and the capacitor.

\subsection{.OP: Circuit when $t < 0 \:s$}

In the table (\ref{tab:SimNodMet}), we display all the branch currents and
node voltages obtained from the simulation of the circuit ilustrated in figure (\ref{fig:2}). 
We also present, side by side, the table from section (\ref{se:2.1}) to facilitate comparing results.

\begin{table}[H]
    \caption{Nodal Method Analysis - $t < 0 \: s$}
    \begin{minipage}{.5\linewidth}

        \centering
        \caption{Theoretical Results}
        \begin{tabular}{|l|r|}
            \hline    
            {\bf Quantity} & {\bf Value} \\ \hline
            \input{Teo1-NodalMethod}
        \end{tabular}

    \end{minipage}%
    \begin{minipage}{.5\linewidth}

        \centering
        \caption{Simulated Results}
        \begin{tabular}{|l|r|}
            \hline    
            {\bf Quantity} & {\bf Value [A or V]} \\ \hline
            \input{op1_tab}
        \end{tabular}

    \end{minipage}
    \label{tab:SimNodMet}
\end{table}

Comparing the values derived from the theoretical analysis with the simulated ones, we
can observe that they are very similar, only slightly differing from each other probably
due to the program's precision.


\subsection{.OP: Circuit when $t = 0^{+} \:s$}

In the table below, (\ref{tab:SimEqRes}), we display all the branch currents and
node voltages obtained from the simulation of the circuit ilustrated in figure (\ref{fig:4}).
The need for this step is explained in the subsection (\ref{sec:AnalEqRes}) and in this subsection
we look to confirm the values previously obtained with the theoretical analysis. We once again compare both 
theoretical results and simulation results side by side.

\begin{table}[H]
    \caption{Nodal Method Analysis - $t = 0^{+} \: s$}
    \begin{minipage}{.5\linewidth}

        \centering
        \caption{Theoretical Results}
        \begin{tabular}{|l|r|}
            \hline    
            {\bf Quantity} & {\bf Value} \\ \hline
            \input{Teo2-EquivalentResistance}
        \end{tabular}

    \end{minipage}%
    \begin{minipage}{.5\linewidth}

        \centering
        \caption{Simulated Results}
        \begin{tabular}{|l|r|}
            \hline
            {\bf Quantity} & {\bf Value [A or V]} \\ \hline
            \input{op2_tab}
        \end{tabular}

    \end{minipage}
    \label{tab:SimEqRes}
\end{table}

Similarly to the previous subsection, we can observe that the values from the theoretical analysis
and the simulated ones are very similar, only slightly differing from each other due to the
program's precision.


\subsection{.TRAN: Natural Solution of the Circuit}

In figure (\ref{fig:SimNatSol}), we ilustrate the natural solution of the circuit in node 6 in
the interval $t \in [0,20] \: ms$. We can notice both graphs obtained from the theoretical and simulated
analysis look pretty much the same as expected.

\begin{figure}[H] \centering
    \includegraphics[width=0.6\linewidth]{trans3.pdf}
    \caption{Natural Solution of the Voltage at Node 6, $v_{6n}(t)$}
    \label{fig:SimNatSol}
\end{figure}



\subsection{.TRAN: Total Solution of the Circuit}

In figure (\ref{fig:SimTotSol}), we display the first $20 \:ms$ of the voltage in node 6 of the circuit, colored green,
when subjected to the stimulus, colored red. Once again, both graphs obtained from the theoretical
and simulated analysis look the same as expected.

\begin{figure}[H] \centering
    \includegraphics[width=0.6\linewidth]{trans4.pdf}
    \caption{Total Solution: \textcolor{red}{Red - Voltage Source}; \textcolor{green}{Green - Node 6}}
    \label{fig:SimTotSol}
\end{figure}



\subsection{.AC: Circuit Frequency Response}

In this section, we study the frequency response obtained from an
AC analysis between $0.1 Hz$ and $1 MHz$, similar to what was done in section (\ref{sec:AnalFRes}).
\\
In figure (\ref{fig:FrequencyResponseSim}), we show the magnitudes in decibels (dB) and the phases in degrees
of the input source voltage (red), of the capacitor
voltage (green) and of the voltage in node 6 (blue), as a function of the frequency, which is displayed logarithmically
in the x axis.

\begin{figure}[H]
    \centering
    \includegraphics[scale=0.35]{trans5.pdf}
    \includegraphics[scale=0.35]{trans6.pdf}
    \caption{Frequency Response - Magnitude and Phase: \textcolor{red}{Red - Voltage Source}; \textcolor{green}{Green - Node 6}; \textcolor{blue}{Blue - Capacitor}}
    \label{fig:FrequencyResponseSim}
\end{figure}

By comparing the graphs obtained by simulated and theoretical means, we can once again come to the conclusion that
both methods produce the same overall results, never producing a noticeable discrepancy.
\\
Therefore, because the plots are basically the same, the reasons on why the responses of the elements plotted differ
from each other can be found in subsection (\ref{sec:AnalFRes}).